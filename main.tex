\documentclass[aspectratio=169,fleqn]{beamer}
\PassOptionsToPackage{english}{babel}
\usepackage{standardslides}

\usepackage{svg}

\usepackage{tikz}
\usepackage{pifont}
\usepackage{listings}
\usepackage{colortbl}
\newlength{\listingframemargin}
\setlength{\listingframemargin}{1em}
\newlength{\listingmargin}
\setlength{\listingmargin}{0.08\textwidth}

\definecolor{codeDarkGray}{gray}{0.2}
\definecolor{codeGray}{gray}{0.4}
\definecolor{codeLightGray}{rgb}{0.94,0.94,0.91}
\definecolor{codeBorder}{rgb}{0.34,0.24,0.21}
\definecolor{MidnightBlue}{rgb}{0.1, 0.1, 0.8}

\lstdefinestyle{standard}{%
  belowcaptionskip=0.5\baselineskip,
  breaklines=true,
  frameround=tttt,
  % frame=false,
  xleftmargin=0em,
  xrightmargin=0em,
  showstringspaces=false,
  showtabs=false,
  % tab=\smash{\rule[-.2\baselineskip]{.4pt}{\baselineskip}\kern.5em},
  basicstyle= \fontfamily{pcr}\selectfont\tiny\bfseries,
  keywordstyle= \bfseries\color{MidnightBlue}, %\color{codeDarkGray},
  commentstyle= \itshape\color{codeGray},
  identifierstyle=\color{codeDarkGray},
  stringstyle=\color{BurntOrange}, %\color{codeDarkGray},
  numberstyle=\tiny\ttfamily,
  % numbers=left,
  numbersep = 1em,
  % stepnumber = 1,
  % captionpos=t,
  tabsize=2,
  % backgroundcolor=\color{codebLightGray},
  rulecolor=\color{codeBorder},
  framexleftmargin=\listingframemargin,
  framexrightmargin=\listingframemargin
}

\newcommand{\inputCodeBlock}[1]{%
  % \begin{mybox}
    \begin{center}
      % \begin{minipage}[c]{0.7\textwidth}
        \lstinputlisting[%
          style = standard,
          language = c++,
          morekeywords={constexpr,noexcept,decltype,size_t,uint32_t,uint64_t,__m256i,__m256,__m256d,__m128i,__m128,__m128d}
        ]{#1}
      % \end{minipage}
    \end{center}
  % \end{mybox}
}

\def\UrlBigBreaks{\do\/\do-\do:}

\setbeamertemplate{background}{
  \includegraphics[width=\paperwidth,height=\paperheight]{images/background-tessellation.png}
}

\setbeamertemplate{footline}[frame number]
\setbeamertemplate{navigation symbols}{}

\title{%
  Algorithmical Geometry: \\ Computation of Delaunay Triangulations \\ Using a Divide-and-Conquer Algorithm%
}
% \subtitle{Master's Thesis Defense and Presentation}
\author{Markus Pawellek}

\bibliography{references}

\begin{document}

\selectlanguage{english}

%{ % all template changes are local to this group.
%  \setbeamertemplate{navigation symbols}{}
%  \begin{frame}<article:0>[plain]
%    \begin{tikzpicture}[remember picture,overlay]
%      \node[at=(current page.center)] {
%        \includegraphics[keepaspectratio,
%                         width=1.2\paperwidth,
%                         height=\paperheight]{images/banner.png}
%      };
%    \end{tikzpicture}
%  \end{frame}
%}

\begin{frame}[plain]{}
  \centering
  \includegraphics[height=\textheight]{images/lenna.png}
\end{frame}

\begin{frame}[plain]{}
  \centering
  \includegraphics[height=\textheight,trim={10px 10 10 45},clip]{images/lenna-tessellation.png}
\end{frame}

\frame[plain]{\titlepage}
\begin{frame}[plain]{Outline}
  \footnotesize
  \hfill\parbox[t][7cm][l]{0.9\textwidth}{\tableofcontents}
\end{frame}

\setcounter{framenumber}{0}

\section{Related Work}
  \begin{frame}{Related Work}
    % !!!Here should be an image of a wireframe for a mesh or a 3D grid for FEM simulation.
    % \bigskip
    %\begin{minipage}[c]{0.49\textwidth}
    %  \begin{figure}
    %    \centering
        %\includesvg[width=0.9\textwidth]{images/mesh-generation.svg}
    %    \includegraphics[width=0.9\textwidth,trim={10px 10 10 45},clip]{images/mosaic.png}
    %  \end{figure}
      %{%
      %  \fontsize{4}{5}\selectfont%
      %  *\url{https://upload.wikimedia.org/wikipedia/commons/b/b8/Approx-3tori.svg}, December 29, 2021%
      %}
    %\end{minipage}
    %\hfill
    %\begin{minipage}[c]{0.45\textwidth}
    %  \pause
      Educational Problems:
      \begin{itemize}
        \pause
        \item Many Resources
        \pause
        \item Duality to Voronoi Diagrams
        \pause
        \item%
          Multiple Algorithm Types: \\
          Incremental, Sweepline, Divide-and-Conquer
        \pause
        \item Varying Data Structures
      \end{itemize}
    %\end{minipage}
  \end{frame}

  \begin{frame}{Related Work: References}
    \small
    \onslide<+->
    \begin{description}
      % This custom command does not work...
      % \newcommand\mycommand[1]{\item[\autocite{#1}] \citeauthor{#1}, \citetitle{#1}, \citeyear{#1}}
      \item<+->[\citeyear{lee1980}] \citeauthor{lee1980}, \citetitle{lee1980}
      \item<+->[\citeyear{guibas1985}] \textbf<7>{\citeauthor{guibas1985}, \citetitle{guibas1985}}
      \item<+->[\citeyear{dwyer1987}] \citeauthor{dwyer1987}, \citetitle{dwyer1987}
      \item<+->[\citeyear{shewchuk1996}] \citeauthor{shewchuk1996}, \citetitle{shewchuk1996}
      \item<+->[\citeyear{fuetterling2014}] \citeauthor{fuetterling2014}, \citetitle{fuetterling2014}
    \end{description}
  \end{frame}

\section{Mathematical Preliminaries}
  \begin{frame}{Mathematical Preliminaries: Triangle and Circumcircle}
    % Definition of a triangle can be difficult.
    % Hence, we use and indirect approach.
    % \onslide<+->
    \begin{minipage}[c]{0.45\textwidth}
      \textbf{Triangle} \\
      $A, B, C \in \setReal^2$ affinely independent \\
      define vertices of a triangle.

      \bigskip

      \pause
      \textbf{Circumcircle}\\
      Circle that intersects with \\
      all vertices of the triangle.
    \end{minipage}
    \hfill
    \begin{minipage}[c]{0.49\textwidth}
      \centering
      \onslide<1->
      \only<1>{\includegraphics[width=0.9\textwidth]{figures/triangle.pdf}}%
      \only<2>{\includegraphics[width=0.9\textwidth]{figures/triangle-circumcirle.pdf}}
    \end{minipage}
  \end{frame}

  \begin{frame}{Mathematical Preliminaries: %
    \only<1>{Point Set}%
    \only<2>{Triangulation}%
    \only<3-5>{Delaunay Triangulation}%
  }
    \begin{minipage}[c]{0.45\textwidth}
      \onslide<+->
      \textbf{Point Set}\\
      $\mathscr{V}\subset\setReal^2$ finite, $\#\mathscr{V}\geq 3$, \\
      affinely span $\setReal^2$

      \bigskip

      \onslide<+->
      \textbf{Triangulation}\\
      Planar straight-line graph over $\mathscr{V}$ \\
      such that its edges form a maximal subset of non-crossing edges.

      \bigskip

      \onslide<+->
      \textbf{Delaunay Triangulation}\\
      Circumcircle of any triangle \\
      contains no other points of $\mathscr{V}$.
    \end{minipage}
    \hfill
    \begin{minipage}[c]{0.49\textwidth}
      \centering
      \onslide<1->
      \only<1>{\includegraphics[width=\textwidth]{figures/point-set.pdf}}%
      \only<2>{\includegraphics[width=\textwidth]{figures/triangulation.pdf}}%
      \only<3>{\includegraphics[width=\textwidth]{figures/triangulation-circumcircle.pdf}}%
      \only<4>{\includegraphics[width=\textwidth]{figures/delaunay-triangulation.pdf}}%
      \only<5>{\includegraphics[width=\textwidth]{figures/delaunay-triangulation-circumcircle.pdf}}%
    \end{minipage}
  \end{frame}

  \begin{frame}{Mathematical Preliminaries: Properties}
    \begin{minipage}[c]{0.49\textwidth}
      \center
      \only<1-4>{\includegraphics[width=\textwidth]{figures/delaunay-triangulation.pdf}}%
      \only<5>{\includegraphics[width=\textwidth]{figures/delaunay-triangulation-convex-hull.pdf}}%
      \only<6>{\includegraphics[width=\textwidth]{figures/delaunay-triangulation-voronoi.pdf}}%
    \end{minipage}
    \hfill
    \begin{minipage}[c]{0.49\textwidth}
      \begin{itemize}
        \item<2-> Existence is guaranteed
        \item<3-> Unique if there are no four points that are cocircular
        \item<4-> Optimality: maximization of the minimum angle of all angles
        \item<5-> Convex hull is contained
        \item<6-> Dual of Voronoi diagram
      \end{itemize}
    \end{minipage}
  \end{frame}

\section{Geometric Primitives}
  \begin{frame}{Geometric Primitives: Triangle Orientation}
    \begin{minipage}[c]{0.4\textwidth}
      %\begin{figure}
        \centering
        \includegraphics{figures/triangle-counterclockwise.pdf}
      %\end{figure}
    \end{minipage}
    \hfill
    $\longleftrightarrow$
    \hfill
    \begin{minipage}[c]{0.4\textwidth}
      %\begin{figure}
        \centering
        \includegraphics{figures/triangle-clockwise.pdf}
      %\end{figure}
    \end{minipage}

    \pause
    Counterclockwise Orientation
    $\quad\iff\quad$ $C$ is left of $\overline{AB}$

    \bigskip

    \pause
    \begin{mybox}
    \[
      0 <
      \begin{vmatrix}
        A_x & A_y & 1 \\
        B_x & B_y & 1 \\
        C_x & C_y & 1 \\
      \end{vmatrix}
      \pause
      =
      \begin{vmatrix}
        B_x - A_x & B_y - A_y \\
        C_x - A_x & C_y - A_y \\
      \end{vmatrix}
      \pause
      =
      \det
      \begin{pmatrix}
        B-A && C-A
      \end{pmatrix}
    \]
    \end{mybox}
  \end{frame}

  \begin{frame}{Geometric Primitives: Inside Circumcircle}
    \pause
    \begin{minipage}[c]{0.49\textwidth}
      \begin{mybox}
        \[
          0 <
          \begin{vmatrix}
            A_x & A_y & A_x^2 + A_y^2 & 1 \\
            B_x & B_y & B_x^2 + B_y^2 & 1 \\
            C_x & C_y & C_x^2 + C_y^2 & 1 \\
            D_x & D_y & D_x^2 + D_y^2 & 1 \\
          \end{vmatrix}
        \]
        \pause
        \[
          =
          \begin{aligned}[t]
            &\scalarProduct{x}{\mathrm{adj}\begin{pmatrix}u & v \end{pmatrix}^\mathrm{T} \begin{pmatrix}\norm{u}^2 \\ \norm{v}^2 \end{pmatrix}} \\
            &- \det\begin{pmatrix}u & v \end{pmatrix}\norm{x}^2
          \end{aligned}
        \]
      \end{mybox}
      $u \define B-A ,\quad v \define C-A ,\quad x \define D-A$
    \end{minipage}
    \hfill
    \begin{minipage}[c]{0.49\textwidth}
      \center
      \onslide<1->
      \includegraphics[width=0.9\textwidth]{figures/inside-circumcircle.pdf}
    \end{minipage}
  \end{frame}

\section{Quad-Edge Data Structure}
  \begin{frame}{Quad-Edge Data Structure: Scheme}
    \begin{minipage}[c]{0.49\textwidth}
      Edge-Based List-Like Data Structure \\
      for Storing Neighbor Information:
      \onslide<+->
      \begin{itemize}
        \item<+-> Directed edges for vertices
        \item<+-> Pointer to ccw. next directed edge with same origin vertex
        \item<+-> Directed dual edges for adjacent faces
        \item<+-> Pointer to ccw. next directed dual edge with same origin face
      \end{itemize}
    \end{minipage}
    \hfill
    \begin{minipage}[c]{0.49\textwidth}
      \center
      \only<1>{\includegraphics[width=0.9\textwidth]{figures/quad-edge-empty.pdf}}%
      \only<2-3>{\includegraphics[width=0.9\textwidth]{figures/quad-edge.pdf}}%
      \only<4-5>{\includegraphics[width=0.9\textwidth]{figures/quad-edge-dual.pdf}}%
      \only<6->{\includegraphics[width=0.9\textwidth]{figures/quad-edge-both.pdf}}%
    \end{minipage}
  \end{frame}

  \begin{frame}{Quad-Edge Data Structure: Implementation}
    \begin{minipage}[c]{0.49\textwidth}
      %\onslide<1>
      \only<1>{%
        \begin{figure}
          \includegraphics[width=\textwidth]{figures/quad-edge-struct.pdf}
        \end{figure}%
        \begin{mybox}
          \center
          \vspace{-1em}
          \inputCodeBlock{listings/quad-edge-algebra.cpp}
        \end{mybox}%
      }%
      %\onslide<2>
      \only<2>{%
        %\begin{mybox}
        %  \inputCodeBlock{listings/quad-edge-algebra-operations.cpp}
        %\end{mybox}}%
        \begin{figure}
          \includegraphics[width=\textwidth]{figures/quad-edge-struct-rot.pdf}
        \end{figure}%
        \begin{mybox}
          \[
            \function{\mathrm{rot}}{\setNatural_0}{\setNatural_0}
          \]
          \[
            \mathrm{rot}(x) = 4 \cdot \floorBrackets{\frac{x}{4}} + (x+1 \mod 4)
          \]
        \end{mybox}%
      }%
    \end{minipage}
    \hfill
    \begin{minipage}[c]{0.49\textwidth}
      \center
      \only<1>{\includegraphics[width=0.9\textwidth]{figures/quad-edge-both.pdf}}%
      \only<2>{\includegraphics[width=0.9\textwidth]{figures/quad-edge-rot.pdf}}%
    \end{minipage}
  \end{frame}

  \begin{frame}{Quad-Edge Data Structure: Edge Functions and Operators}
    \begin{figure}
      \center
      \includegraphics[height=0.6\textheight]{figures/quad-edge-vertex-functions.pdf}
      \hspace{2em}
      \includegraphics[height=0.6\textheight]{figures/quad-edge-face-functions.pdf}
    \end{figure}
    \onslide<+->
    \begin{itemize}
      \item<+-> Create a new edge
      \item<+-> Delete existing edge
      \item<+-> Connect points by a new edge
    \end{itemize}
  \end{frame}

\section{Algorithm}
  \begin{frame}{Algorithm: Overview}
    %\only<1>{%
      \onslide<+->
      \onslide<+->
      \begin{mybox}
        \textbf{Triangulation Algorithm}
        \begin{enumerate}
          \item<+-> Sort the given point set by increasing $x$ coordinate.
          \item<+-> Triangulate sorted point set.
        \end{enumerate}
      \end{mybox}%
      \bigskip
      %\begin{itemize}
      %  \item Preprocessing Step with Complexity $\mathscr{O}(n\log n)$
      %  \item Sorting of points makes split constant time operation
      %\end{itemize}%
    %}%
    %\only<2>{%
    \bigskip
      \onslide<+->
      \begin{mybox}
        \textbf{Subroutine: Triangulate}
        \begin{enumerate}
          \item<+-> If point count is smaller than four, make edge or triangle and return.
          %\item If $\#\mathscr{V}<4$ then make $\mathscr{T}(\mathscr{V})$ an edge or a triangle and break
          %\item $\mathbf{if}\ \#\mathscr{V}<4 \ \mathbf{then}$ \\ $\quad \mathscr{T}(\mathscr{V}) \longleftarrow \mathrm{edge\_or\_triangle}(\mathscr{V})$ \\ $\quad \mathbf{return}$
            %\begin{enumerate}
            %  \item $\mathscr{T}(\mathscr{V}) \longleftarrow \mathrm{edge or triangle}(\mathscr{V})$
            %  \item $\mathrm{break}$
            %\end{enumerate}
          %\item Separate $\mathscr{V}$ into halves $\mathscr{L}$ and $\mathscr{R}$
          \item<+-> Split point set into left and right half.
          \item<+-> Triangulate left and right half.
          \item<+-> Merge left and right triangulations.
          %\item $(\mathscr{L},\mathscr{R}) \longleftarrow \mathrm{split}(\mathscr{V})$
          %\item Triangulate $\mathscr{L}$ and $\mathscr{R}$ into $\mathscr{T}(\mathscr{L})$ and $\mathscr{T}(\mathscr{R})$
          %\item $(\mathscr{T}(\mathscr{L}), \mathscr{T}(\mathscr{R})) \longleftarrow (\mathrm{triangulate}(\mathscr{L}),\mathrm{triangulate}(\mathscr{R}))$
          %\item $\mathscr{T}(\mathscr{L}) \longleftarrow \mathrm{triangulate}(\mathscr{L})$
          %\item $\mathscr{T}(\mathscr{R}) \longleftarrow \mathrm{triangulate}(\mathscr{R})$
          %\item Merge $\mathscr{T}(\mathscr{L})$ and $\mathscr{T}(\mathscr{R})$
          %\item $\mathscr{T}(\mathscr{V}) \longleftarrow \mathrm{merge}(\mathscr{T}(\mathscr{L}), \mathscr{T}(\mathscr{R}))$
        \end{enumerate}
      \end{mybox}%
      %\begin{itemize}
      %  \item Divide-and-Conquer Algorithm
      %  \item Key is merge step
      %\end{itemize}%
    %}%
  \end{frame}

  \begin{frame}{Algorithm: Merge Triangulations Example}
    \only<+>{\includegraphics[width=\textwidth]{figures/merge-empty.pdf}}%
    \only<+>{\includegraphics[width=\textwidth]{figures/merge-lct.pdf}}%
    \only<+>{\includegraphics[width=\textwidth]{figures/merge-invalid-edge-circle-test.pdf}}%
    \only<+>{\includegraphics[width=\textwidth]{figures/merge-invalid-edge-removal.pdf}}%
    \only<+>{\includegraphics[width=\textwidth]{figures/merge-invalid-edge-circle-test-right.pdf}}%
    \only<+>{\includegraphics[width=\textwidth]{figures/merge-cross-edge-circle-test.pdf}}%
    \only<+>{\includegraphics[width=\textwidth]{figures/merge-cross-edge-insertion.pdf}}%
    \only<+>{\includegraphics[width=\textwidth]{figures/merge-second-circle-test.pdf}}%
    \only<+>{\includegraphics[width=\textwidth]{figures/merge-second-cross-edge.pdf}}%
    \only<+>{\includegraphics[width=\textwidth]{figures/merge-uct.pdf}}%
    \only<+>{\includegraphics[width=\textwidth]{figures/merge-finish.pdf}}%
  \end{frame}

  \begin{frame}{Algorithm: Merge Triangulations}
    \onslide<+->
    \onslide<+->
    \begin{mybox}
      \textbf{Subroutine: Merge Triangulations}
      \begin{enumerate}
        %\item $\mathscr{T}(\mathscr{V}) \longleftarrow \mathscr{T}(\mathscr{L}) \cup \mathscr{T}(\mathscr{R})$
        \item<+-> Compute and add lower common tangent.
        %\item Find lower common tangent of $\mathscr{L}$ and $\mathscr{R}$
        %\item $b \longleftarrow \mathrm{lower\_common\_tangent}(\mathscr{L}, \mathscr{R})$
        %\item Add lower common tangent as $b$
        %\item $\mathscr{T}(\mathscr{V}) \longleftarrow \mathscr{T}(\mathscr{V}) \cup b$
        %\item Loop until $b$ is upper common tangent:
        \item<+-> Use lower common tangent as baseline.
        \item<+-> Loop until baseline becomes upper common tangent:
          \begin{enumerate}
            \item<+-> Remove invalid edges adjacent to and above baseline.
            \item<+-> Insert cross edge above baseline.
            \item<+-> Make this cross edge the new baseline.
          \end{enumerate}
        %\item $\mathbf{while} \ b \neq \mathrm{upper\_common\_tangent}(\mathscr{L}, \mathscr{R})$
        %  \begin{enumerate}
        %    \item $\mathscr{T}(\mathscr{V}) \longleftarrow \mathscr{T}(\mathscr{V}) \setminus \mathrm{invalid\_edges}(\mathscr{T}(\mathscr{L}), b)$
        %    \item $\mathscr{T}(\mathscr{V}) \longleftarrow \mathscr{T}(\mathscr{V}) \setminus \mathrm{invalid\_edges}(\mathscr{T}(\mathscr{R}), b)$
        %    %\item Insert cross edge $c$ above $b$
        %    \item $b \longleftarrow \mathrm{cross\_edge}(\mathscr{T}(\mathscr{V}), b)$
        %    \item $\mathscr{T}(\mathscr{V}) \longleftarrow \mathscr{T}(\mathscr{V}) \cup b$
        %  \end{enumerate}
      \end{enumerate}
    \end{mybox}
    \onslide<+->
    \textbf{Details:}
    \begin{itemize}
      %\item linear complexity by using Euler's formula for planar graphs: $v-e+f=2$
      \item<+-> Computation of lower common tangent
      \item<+-> Circle test for adjacent edges
      \item<+-> Circle test for cross edge
    \end{itemize}
  \end{frame}

  %\begin{frame}{Algorithm: Insert Cross Edges}
  %  \begin{mybox}
  %    \textbf{Subroutine: Remove Invalid Edge}
  %    \begin{enumerate}
  %      \item While adjacent edge lies in circumcircle
  %      \item remove edge
  %    \end{enumerate}
  %  \end{mybox}%
  %  \begin{mybox}
  %    \textbf{Subroutine: Insert Cross Edge}
  %    \begin{enumerate}
  %      \item Insert cross edge by using circle test
  %    \end{enumerate}
  %  \end{mybox}%
  %\end{frame}

  %\begin{frame}{Algorithm: Lower Common Tangent}
  %  \begin{mybox}
  %    \textbf{Subroutine: Lower Common Tangent}
  %    \begin{enumerate}
  %      \item $l$ and $r$ as inner convex hull edges
  %      \item Loop:
  %      \begin{enumerate}
  %        \item If $\mathrm{leftof}(\mathrm{o}(r), l)$ then $l \longleftarrow \mathrm{lnext}(l)$
  %        \item If $\mathrm{rightof}(\mathrm{o}(l), r)$ then $r \longleftarrow \mathrm{rprev}(r)$
  %        \item Else break
  %      \end{enumerate}
  %    \end{enumerate}
  %  \end{mybox}
  %\end{frame}

  \begin{frame}{Algorithm: Complexity}
    \onslide<+->
    \begin{minipage}[c]{0.4\textwidth}
      \begin{itemize}
        \item<+-> $n\in\setNatural$ points
        \item<+-> $T(n)$ runtime
        \item<+-> Use master theorem
        \item<+-> Complexity of merge is linear in worst case
      \end{itemize}
      \medskip
      \onslide<+->
      \[
        T(n) = 2 T\roundBrackets{\frac{n}{2}} + \mathscr{O}(n)
      \]
      \onslide<+->
      \begin{mybox}
        \[
          T(n) = \mathscr{O}(n\log n)
        \]
      \end{mybox}
    \end{minipage}
    \hfill
    \onslide<1->
    \begin{minipage}[c]{0.49\textwidth}
      \begin{mybox}
        \textbf{Subroutine: Triangulate}
        \begin{enumerate}
          \item If $n\leq 3$, make edge or triangle and return.
          \item Split point set into left and right.
          \item Triangulate left and right half.
          \item Merge left and right triangulations.
        \end{enumerate}
      \end{mybox}%
    \end{minipage}
  \end{frame}

  \begin{frame}{Algorithm: Correctness}
    %Proof by induction: \\
    %Show that for two given Delaunay triangulations $\mathscr{T}(\mathscr{L})$ and $\mathscr{T}(\mathscr{R})$ separated by a vertical line the merge subroutine generates a Delaunay triangulation $\mathscr{T}(\mathscr{L}\cup\mathscr{R})$.
    \onslide<+->
    \begin{itemize}
      \item<+-> Proof by induction for merging left and right half
      \item<+-> For two Delaunay triangulations separated by a vertical line, it is enough to remove inner edges and insert cross edges.
      \item<+-> Common tangents are elements of the Delaunay triangulation.
      \item<+-> Removed edges are indeed not Delaunay.
      \item<+-> Insertion of cross edges generates new Delaunay triangle.
      \item<+-> There are no other edges that have to be removed.
      \item<+-> Algorithm is correct
    \end{itemize}
  \end{frame}

\section{Implementation Notes}
  \begin{frame}{Implementation Notes}
    \begin{itemize}
      \item<+-> Geometric primitives need to be robustly computed
      \item<+-> Still no robust split, use Dwyer algorithm instead %(no sorting, parallelization)
      \item<+-> Triangular data structure increases speed but algorithm is more complicated
      \item<+-> Divide-and-conquer variant seems to be most powerful and robust
    \end{itemize}
  \end{frame}

%\section{Applications}

\section{Conclusions}
  \begin{frame}{Conclusions}
    \onslide<+->
    \textbf{Summary:}
    \begin{itemize}
      \item<+-> Delaunay triangulation can be generated by given divide-and-conquer algorithm in $\mathscr{O}(n\log n)$
      \item<+-> Data structure needs to store neighbor information
    \end{itemize}
    \bigskip
    \onslide<+->
    \textbf{Future Work:}
    \begin{itemize}
      \item<+-> Use triangular data structure instead of quad-edge data structure
      \item<+-> Use \citeauthor{dwyer1987}'s approach to make algorithm more robust
      \item<+-> Parallelization
      \item<+-> Multi-dimensional Implementation
    \end{itemize}
  \end{frame}


\setcounter{backupcounter}{\value{framenumber}}

\begin{frame}[plain]
  \vfill
  \centering
  \begin{beamercolorbox}[sep=8pt,center,shadow=true,rounded=true]{title}
    \usebeamerfont{title}%
    Thank you for Your Attention!%
    \par%
  \end{beamercolorbox}
  \vfill
\end{frame}

\begin{frame}[plain]
  \frametitle{References}
  % \tiny
  \AtNextBibliography{\tiny}
  \begin{multicols}{2}
    \nocite{*}
    \printbibliography
  \end{multicols}
\end{frame}

\setcounter{framenumber}{\value{backupcounter}}

\end{document}
